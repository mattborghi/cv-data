%%%%%%%%%%%%%%%%%%%%%%%%%%%%%%%%%%%%%%%%%
% Twenty Seconds Resume/CV
% LaTeX Template
% Version 1.1 (8/1/17)
%
% This template has been downloaded from:
% http://www.LaTeXTemplates.com
%
% Original author:
% Carmine Spagnuolo (cspagnuolo@unisa.it) with major modifications by 
% Vel (vel@LaTeXTemplates.com)
%
% License:
% The MIT License (see included LICENSE file)
%
%%%%%%%%%%%%%%%%%%%%%%%%%%%%%%%%%%%%%%%%%

%----------------------------------------------------------------------------------------
%	PACKAGES AND OTHER DOCUMENT CONFIGURATIONS
%----------------------------------------------------------------------------------------

\documentclass[letterpaper]{twentysecondcv} 
% a4paper for A4
% letterpaper for letter

%----------------------------------------------------------------------------------------
%	 PERSONAL INFORMATION
%----------------------------------------------------------------------------------------

% If you don't need one or more of the below, just remove the content leaving the command, e.g. \cvnumberphone{}



\cvname{Matias Pablo \newline Borghi Orue} % Your name
\cvjobtitle{MSc. Physics} % Job title/career

\cvdate{12 August 1991} % Date of birth
\cvaddress{borghi.matias@gmail.com} % Short address/location, use \newline if more than 1 line is required
\cvnumberphone{+54 9 11 2157 6504} % Phone number
\cvsite{linkedin.com/in/borghimatias/} % Personal website
\cvgithub{github.com/mattborghi}

%----------------------------------------------------------------------------------------

\begin{document}

%----------------------------------------------------------------------------------------
%	 Education
%----------------------------------------------------------------------------------------

\Education{ \textbf{MSc. Science in Physics} | \newline School of Exact and Natural Sciences \newline National University of La Plata | 2017
} % To have no Education section, just remove all the text and leave \Education{}

%----------------------------------------------------------------------------------------
%	 SKILLS
%----------------------------------------------------------------------------------------

% Skill bar section, each skill must have a value between 0 an 6 (float)
\skills{
\textbf{Technical}: \\ 
 - Python, R, Fortran, C\#, Julia \\
 - HTML, CSS, JavaScript, React, GraphQL, Node.js \\
 - \LaTeX, Git \\ 
\textbf{Personal}: \\
- \textit{Teamwork}: Given my work experience I think teamwork is the best tool you can have to get the results you want to achieve. \\ 
- \textit{Self management}: Given different tasks to developed on my own I achieved beyond expectations. \\ 
- \textit{Willingness to learn}: Newer tasks are frequently developed without prior knowledge but that is not a barrier. \\
- \textit{Problem solving}: How to solve a problem should not be a problem. We should be focusing in getting the best approaches.
}



\Hobbies{ 
- Playing Violin \\ 
- Learning French \\ 
- Playing Football \\
- Cycling
}

\Languages{
- Spanish: Native \\
- English: Fluent \\
- French: Intermediate \\
}

%------------------------------------------------

% Skill text section, each skill must have a value between 0 an 6


%----------------------------------------------------------------------------------------

\makeprofile % Print the sidebar

%----------------------------------------------------------------------------------------
%	 INTERESTS
%----------------------------------------------------------------------------------------


%----------------------------------------------------------------------------------------
%	 EDUCATION
%----------------------------------------------------------------------------------------

\section{Work Experience}

\begin{twenty} % Environment for a list with descriptions
	\twentyitem{Since Mar'19}{Sr. Quantitative Analyst}{Crisil Irevna Argentina S.A., Buenos Aires, Argentina}{Research and development of software for pricing and risk management Equity and Hybrid exotic financial derivatives using Monte Carlo engine. Our stack includes back-end technologies such as Python and Julia as well as front-end technologies like React.js and others while communicating through GraphQL APIs.}
	\twentyitem{Aug17' - \\Feb'19}{Quantitative Analyst}{Crisil Irevna Argentina S.A., Buenos Aires, Argentina}{Pricing and Risk Management Equity and Hybrid (IR/FX/COMM) exotic financial derivative models for a Tier-1 US Investment Bank. \\ Responsible for creating technical documentation and generating executive summary reports in \LaTeX. Creation and execution of Benchmark, Limiting Cases and Stability tests, among others, were also coded in C\#.}
	\twentyitem{Sep'15 -\\ Sep'17}{Teaching Assistant}{National University of La Plata, Buenos Aires, Argentina}{Responsible for teaching fundamental physical concepts such as Classical Mechanics and Electromagnetism to undergraduate students}
    %\twentyitem{<dates>}{<title>}{<location>}{<description>}
\end{twenty}


%----------------------------------------------------------------------------------------
%	 PUBLICATIONS
%----------------------------------------------------------------------------------------

\section{Research and Projects}

\begin{twenty} % Environment for a short list with no descriptions
	\twentyitem{Jul'17}{MSc. Physics Thesis}{}{Master thesis final project titled \texit{Study of phase transitions of an Ising-type model with spin oriented dependent interaction parameter}. The software was coded in Fortran.}
	\twentyitem{Dec'14}{Kerr Black Holes}{}{Work done as the final project of the course entitled Introduction to General Relativity about Kerr Black Holes.}
	\twentyitem{Nov'14}{ALAMBRE Project}{}{Final project done for the course Computer Simulations. It consists in simulating both the High Energy Cosmic Rays and the 30m radio telescope located in the IAR(Argentine Institute of Radio Astronomy), Buenos Aires, using Monte Carlo methods with the final goal set to determine if the antenna could detect the radio emission from the cascades.}
	\twentyitem{Jan-Nov'09}{Hexapod Robot}{}{This project was done during 2009 for my high school's science fair with my brother. It consisted in creating the whole robot chassis and simulating its movements for each one of the three motors for the whole six legs. The software was coded in Basic.}
	%\twentyitemshort{<dates>}{<title/description>}
\end{twenty}

%----------------------------------------------------------------------------------------
%	 AWARDS
%----------------------------------------------------------------------------------------

\section{Achievements}

\begin{twenty} % Environment for a short list with no descriptions
	\twentyitem{Oct'09}{Best High School Projects}{}{Second place for best high school projects at National University of La Matanza}
	\twentyitem{Feb'09}{Best High School GPAs}{}{Received scholarship for having one of the best GPAs in high school.}
	%\twentyitemshort{<dates>}{<title/description>}
\end{twenty}

%----------------------------------------------------------------------------------------
%	 EXPERIENCE
%----------------------------------------------------------------------------------------

% \section{Electives and MOOCs}

% \begin{twenty} % Environment for a list with descriptions

% 	\twentyitem{Electives}{Reverse Engineering \& Rapid Prototyping, Renewable Energy, Quality Control Assurance \& Reliability, Project Appraisal, Public Policy,\\Principles of Management, HR Development, Fundamentals of\\Finance \& Accounting, Srimad Bhagavad Gita}{}{}
% 	\twentyitem{MOOCs}{\textbf{Introduction to R,SQL \& Python Courses on Datacamp}\\Currently pursuing Deep Learning Specialization on Coursera}{}{}
	
% 	%\twentyitem{<dates>}{<title>}{<location>}{<description>}
% \end{twenty}

%----------------------------------------------------------------------------------------
%	 OTHER INFORMATION
%----------------------------------------------------------------------------------------




%----------------------------------------------------------------------------------------
%	 SECOND PAGE EXAMPLE
%----------------------------------------------------------------------------------------

%\newpage % Start a new page

%\makeprofile % Print the sidebar

%\section{Other information}

%\subsection{Review}

%Alice approaches Wonderland as an anthropologist, but maintains a strong sense of noblesse oblige that comes with her class status. She has confidence in her social position, education, and the Victorian virtue of good manners. Alice has a feeling of entitlement, particularly when comparing herself to Mabel, whom she declares has a ``poky little house," and no toys. Additionally, she flaunts her limited information base with anyone who will listen and becomes increasingly obsessed with the importance of good manners as she deals with the rude creatures of Wonderland. Alice maintains a superior attitude and behaves with solicitous indulgence toward those she believes are less privileged.

%\section{Other information}

%\subsection{Review}

%Alice approaches Wonderland as an anthropologist, but maintains a strong sense of noblesse oblige that comes with her class status. She has confidence in her social position, education, and the Victorian virtue of good manners. Alice has a feeling of entitlement, particularly when comparing herself to Mabel, whom she declares has a ``poky little house," and no toys. Additionally, she flaunts her limited information base with anyone who will listen and becomes increasingly obsessed with the importance of good manners as she deals with the rude creatures of Wonderland. Alice maintains a superior attitude and behaves with solicitous indulgence toward those she believes are less privileged.

%----------------------------------------------------------------------------------------

\end{document} 
