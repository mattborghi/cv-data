%-------------------------------------------------------------------------------
%	SECTION TITLE
%-------------------------------------------------------------------------------
\cvsection{Projects {\large \href{https://mattborghi.github.io/projects/}{\faExternalLink}}}


%-------------------------------------------------------------------------------
%	CONTENT
%-------------------------------------------------------------------------------
\begin{cventries}

%---------------------------------------------------------
  \projectentry
    {} % Role
    {
    Background Tasks Queue 
    \href{https://mattborghi.github.io/projects/Background\%20Task\%20Queues/}{\faGithub}
    } % Event
    {} % Location
    {Jan. 2021} % Date(s)
    {
      \begin{cvitems} % Description(s)
          \item {Project implementing a background tasks/jobs queue with React as frontend, Python (Django) to process the requests and Julia to run the jobs communicating using RabbitMQ.}
          \item{Deployed using Heroku, GitHub pages and CloudAMQP.}
      \end{cvitems}
    }

%---------------------------------------------------------
  \projectentry
    {} % Role
    {
    Julia Editor on the Web 
    \href{https://mattborghi.github.io/projects/Julia\%20Editor\%20on\%20the\%20Web/}{\faGithub}
    } % Event
    {} % Location
    {Feb. 2021} % Date(s)
    {
      \begin{cvitems} % Description(s)
        \item{A minimal working project that handles Microsoft’s Monaco Editor on the web with Julia’s syntax highlighting and a dedicated Julia Language Server that allows a certain user to have full access to package documentation, suggestion and snippets, among other things.}
        \item{Built using React.js, Node.js and Julia.}
      \end{cvitems}
    }

%---------------------------------------------------------

  \projectentry
    {} % Role
    {
    Julia Terminals on the Web 
    \href{https://mattborghi.github.io/projects/Julia\%20Terminal\%20Web/}{\faGithub}
    } % Event
    {} % Location
    {Mar. 2021} % Date(s)
    {
      \begin{cvitems} % Description(s)
        \item {A minimal working project that handles multiple pseudoterminals running Julia on React. It’s UI design is based on VSCode and Atom IDEs.}
        \item{Built using React.js, Node.js and Julia.}
      \end{cvitems}
    }

%---------------------------------------------------------
\end{cventries}
